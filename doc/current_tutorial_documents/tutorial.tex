\documentclass{article}

\long\def\comment#1{}
\long\def\cmt#1{}
\long\def\todo#1{}
\comment{
    This document is a combination of "A Tutorial Introduction to PEP" and "Tea for survey processing". Both documents have important information regarding the usage of Tea and this document is meant to combine that information into one source. Clean up and correct as you see fit.

-- DV
}

\usepackage{moreverb, epsfig,calc,amsfonts,natbib,url,xspace,listings,ifthen, float}
\newif\ifimputation
\imputationfalse

% Boldface vectors: \xv produces a boldface x, and so on for all of the following:
\def\definevector#1{\expandafter\gdef\csname #1v\endcsname{{\bf #1}\xspace}}
\def\definemathvector#1{\expandafter\gdef\csname #1v\endcsname{\mbox{{\boldmath$\csname #1\endcsname$}}}}
\definevector{b} \definevector{c} \definevector{d}
\definevector{i} \definevector{j} \definevector{k}
\definevector{p}
%u gets special treatment; see below
 \definevector{v} \definevector{w} \definevector{x} \definevector{y} \definevector{z}
\definevector{A} \definevector{B} \definevector{C} \definevector{D}
\definevector{I} \definevector{J} \definevector{K} \definevector{M}
\definevector{Q} \definevector{R} \definevector{S} \definevector{T} \definevector{U} \definevector{V}
\definevector{W} \definevector{X} \definevector{Y} \definevector{Z}
\def\uv{\mbox{{\boldmath$\epsilon$}}} 
\definemathvector{alpha} \definemathvector{beta} \definemathvector{gamma}
\definemathvector{delta} \definemathvector{epsilon} \definemathvector{iota} \definemathvector{mu}
\definemathvector{theta} \definemathvector{sigma} \definemathvector{Sigma}
\def\Xuv{\underbar{\bf X}}

%code listing:
\lstset{columns=fullflexible, basicstyle=\small,
    emph={size\_t,apop\_data,apop\_model,gsl\_vector,gsl\_matrix,gsl\_rng,FILE},emphstyle=\bfseries}
\def\setlistdefaults{\lstset{ showstringspaces=false,%
 basicstyle=\small, language=C, breaklines=true,caption=,label=%
,xleftmargin=.34cm,%
,frameshape=
,frameshape={nnnynnnnn}{nyn}{nnn}{nnnynnnnn}
}
\lstset{columns=fullflexible, basicstyle=\small, emph={size_t,apop_data,apop_model,gsl_vector,gsl_matrix,gsl_rng,FILE,math_fn},emphstyle=\bfseries}
}
\setlistdefaults

\newenvironment{items}{
\setlength{\leftmargini}{0pt}
\begin{itemize}
  \setlength{\itemsep}{3pt}
  \setlength{\parskip}{0pt}
  \setlength{\parsep}{3pt}
}{\end{itemize}}

\renewcommand{\sfdefault}{phv}
\usepackage{times}
\usepackage{epsfig}
\usepackage{latexsym}
\usepackage{setspace}
\usepackage{verbatim}


%I think the Computer Modern teletype is too wide.
\usepackage[T1]{fontenc}
\renewcommand\ttdefault{cmtt}
\def\tab{\phantom{hello.}}

\def\Re{{\mathbb R}}
\def\tighten{ \setlength{\itemsep}{1pt}
    \setlength{\parskip}{0pt}}
\def\adrec{\textsc{AdRec}\xspace}

\newenvironment{key}[1]{ %
  \setlength{\parsep}{0pt} %
\hspace{-0.5cm} \textbf{#1}:\\ %
  \setlength{\parindent}{0pt} %
  \setlength{\parsep}{3pt} %
}{}

\newenvironment{specbit}{\latexhtml{}{\begin{rawhtml}
<div class="spec">
\end{rawhtml}}}
{\latexhtml{}{\begin{rawhtml}
</div>
\end{rawhtml}}}

\newenvironment{rcode}{\latexhtml{}{\begin{rawhtml}
<div class="rcode">
\end{rawhtml}}}
{\latexhtml{}{\begin{rawhtml}
</div>
\end{rawhtml}}}




\begin{document}
\author{U.S. Census Bureau\\Center for Statistical Research and Methodology Division}
\title{Tea for survey processing}
\maketitle
%just an idea: put silhouettes on the front cover

\begin{abstract}
Tea is a system designed to unify and streamline survey processing.  It detects
observations that are missing data, fail consistency checks, or risk the disclosure
of sensitive information, and uses a unified imputation process to amend these issues.
It includes tools for handling data, generating graphs and tables, generating synthetic
data, and other needs of the survey processor or analyst.
\end{abstract}


\section{Overview}
{\sc We intend} to implement the many steps of survey processing withing a single
framework where the analyst interface is common across surveys, the survey process
is summarized in one place, and the code implementing the procedures is internally
well-documented and reasonably easy to maintain.

Raw data is often rife with missing items, logical errors, and sensitive information.
To ignore these issues risks alienating respondents and data users alike, and so data
modification is a necessary part of the production of quality survey and census data.
Tea is a statistical library designed to eradicate these errors via a framework
that is user-friendly, effectively handles Census-sized data sets, and processes
quickly even with relatively sophisticated methods.

This document is a detailed overview of the Tea system, its components, and their usage.

Section \ref{basicssec} is an overview of the basic mechanics of the system, including the
underlying platforms and the the format of the spec file used to describe a survey.
Because Tea uses R as a front-end, Section \ref{rsec} gives a brief overview of using R
for the purposes of querying, subsetting, and summarizing tables.
Section \ref{impsec} goes into greater detail on the workings of the imputation system,
including its basic procedure, the specific models available, and the use of multiple imputation.

If you have this tutorial you should also have a working installation of Tea and 
a sample directory including {\tt demo.spec, demo.R}, and {\tt dc\_pums\_08.CSV}. 
Basic versions of all of the steps described below are already implemented and running,
though the system will evolve and grow as it is applied in new surveys and environments.

\section{System basics}\label{basicssec}
Tea implements a two step process for addressing issues with raw data. The first is to
identify those failures listed above (missing data, logical errors, or sensitive
information), and then, having identified problem data, impute new values
to replace the old ones. Although the term {\em imputation} is typically used only to 
describe filling in missing data, we use it broadly to mean any modification of a 
data item that involves choosing among alternatives, regardless of which of the 
above failures prompted the fill-in. 

The specification of process details is a plain text file, herein called the {\em
spec file}. Based on your inputs to the spec file (which we will explain later as
to what those inputs can be), you then run a script in a user-friendly statistical
computing framework called {\bf R}. This is where the computing (editing, imputation,
etc) takes place. We will explain this in more detail later as well. For now, let's
look closer at the spec file:

\subsection{The spec file}\label{specsec}
The full specification of the various steps of Tea's process, from input of data to final
production of output tables, is specified in a single file, the {\em spec file}. There
are several benefits to such a workflow.  First, because the spec file is separate from
robust programming languages like \textbf{R} or SAS, it is a simpler grammar that is easier to
write and analysts whose areas of expertise are not in programming can customize Tea
and write technical specifications without the assistance of a programmer or extensive
training in a new programming language. Consider the following script that is taken
from the {\tt demo.spec} file (which you can open using Kate or any other text editor):

\begin{verbatim}
database: test.db

input {
        input_file: dc_pums_08.CSV
        output_table: dc_pums
}
\end{verbatim}

In this snippet of the {\tt demo.spec} file, we specified a database to use, an input
file to be parsed, and an output table to write our imputations to. Behind the scenes,
SQL scripts and C functions are being executed. As we will see, other scripts that
are run from the spec file perform more complicated algorithms; however, before we
go through an example of a full spec file, we discuss the environment and systems in
which Tea runs and the processes underlying the spec file.

\subsection{Environment and underlying systems}
Tea is based on three systems: C, R, and SQL.\footnote{C is the successor to B, which
was the successor to BCPL: basic combined programming language. R, a successor to S,
is named after its authors, Robert Gentleman and Ross Ihaka. SQL stands for structured
query language.} Each provides facilities that complement the others:

{\bf SQL} is designed around making fast queries from databases, such as finding all
observations within a given range or category. Any time we need a subset of the data,
we will use SQL to describe and extract it. SQL is a relatively simple language, so
users unfamiliar with it can probably learn the necessary SQL in a few minutes---in
fact, a reader who claims to know no SQL will probably already be able to read and
modify the SQL-language conditions in the {\tt checks} sections below.

Tea stores data using an SQL database. The system queries the database as needed to
provide input to the statistical/mathematical components (R, C function libraries,
etc.).  Currently, Tea is written to support SQLite as its database interface; however
it would be possible to implement other interfaces such as Oracle or MySQL.

Output at each step is also written to the database to be read as input in the next
step. Thus the state of the data can be recorded at each step in the process to allow
for auditing of suspect changes.

Outside the database, to control what happens and do the modeling, the Tea package
consists of roughly 5,000 lines of R and C code.

{\bf R} is a relatively user-friendly system that makes it easy to interact with data
sets and write quick scripts to glue together segments of the survey-processing pipeline.
R is therefore the interactive front-end for Tea. Users will want to get familiar with
the basics of R.  As with SQL, users well-versed in R can use their additional
knowledge to perform analysis beyond the tools provided by Tea.

{\bf C} is the fastest human-usable system available for manipulating matrices, making
draws from distributions, and other basic model manipulations. Most of the numerical
work will be in C. The user is not expected to know any C at all, because R procedures
are provided that do the work of running the underlying C-based procedures.

%\comment{DV: Taking this out for now until we get all the other syntax settled.

%Now that we have a better idea of the environments in which Tea is run, here is 
%an example of a full {\tt spec} file, {\tt demo.spec}:


%\begin{figure}
%\lstset{columns=fullflexible}
%\lstinputlisting{../../demo/demo.spec}
%\caption{A sample spec file that imputes an AGEP variable using WAGP, SEX, 
%and the {\tt hot deck} and {\tt ols} imputation models. The contents of this spec 
%file are explained below}
%\end{figure}

%}

The configuration system ({\tt spec} file) is a major part of the user interface with Tea. 
As is evident from {\tt demo.spec}, there are many components of the spec file that all 
perform certain functions. We begin by explaining the concept of \textit{keys}.

\subsection{Keys}
Everything in the spec file is a key/value pair (or, as may be more familiar to 
you, a tag: data definition). Each key in the spec file has a specific purpose and 
will be outlined in this tutorial.
To begin, we start in the header of {\tt demo.spec}:

\begin{lstlisting}[language=]
database: demo.db
id: SSN
\end{lstlisting}

Here, {\tt database: demo.db} and {\tt id: SSN} are examples of key: value pairs.
As is the case in {\tt demo.spec}, your spec file must begin by providing the
database ({\tt database:your\_database.db}) and the unique identifier ({\tt
id:your\_unique\_identifier}). The {\tt database:} key identifies the database file
where all of your data will be manipulated during the various processes of Tea.
The {\tt id} key provides a column in the database table that serves as the unique
identifier for each set of data points in your data set. Though the {\tt id} key is
not strictly necessary, we strongly advise that you include one in your spec file to
prevent any unnecessary frustration as most of Tea's routines require its use. More
information on both of these keys can be found in the appendix of this tutorial.\\

You may have noticed that keys are assigned values with the following syntax:\\

{\tt key: value}\\

This syntax is equivalent to the following form with curly braces:
\begin{lstlisting}[language=]
database{
	demo.db
}

id {
	SSN
}
\end{lstlisting}

Clearly, this form is not as convenient as the {\tt key: value} form for single values. However, it allows 
us to have multiple values associated with a single line of data, and even subkeys. For example, take  
the next line in {\tt demo.spec} (the computer will 
ignore everything after a \#, so those lines are comments for your fellow humans):

\begin{lstlisting}[language=]
checks { #all of the following values are associated with the "checks" key
	age < 15 && status='married'
	age > 99
}
\end{lstlisting}

The elements in a group don't just have to be values; they can also be key:value pairs:

\begin{lstlisting}[language=]
input { 
    input file: ss08pdc.CSV
    overwrite: no
    output table: dc
}
\end{lstlisting}

%  DV: Here we should explain how subkeys are used (a fact of which I'm not entirely 
%  sure myself). After explaining this we have:

We thus have a main key, {\tt input} and subkeys.  This is useful for keeping the spec
file organized and making sure that keys from one part of the pipeline do not clash with
keys from other parts.  In the database, the keys and subkeys get merged together,
to produce a key:value table that, given the above inputs, would look like this:

\begin{verbatim}
key                 value
------------------  -----------
database            demo.db
id                  SSN
input/input file    ss08pdc.CSV
input/overwrite     no
input/output table  dc
\end{verbatim}

It is worth getting familiar with this internal form, because Tea prints error messages
using this form.  We will discuss this more later in the tutorial when we talk about
running your spec file in R.  Here is a summary of the syntax rules for keys:

\begin{itemize}
\item You can have several values in curly braces, each on a separate line, which are
added to the key/value list. Order is preserved.

\item You can have as many levels of subkeys as you wish.

\item If there is a subkey, then its name is merged with the parent key via a slash.

\item As a shortcut, you can replace {\tt key \{single value\}} with {\tt key:single
value}.

\item Each value takes up exactly one line (unless you explicitly combine lines; see
below). Otherwise, white space is irrelevant to the parser. Because humans will also
read the spec file, you are strongly encouraged to use indentation to indicate group
membership.

\item If you need to continue a value over several lines, put a backslash at the end
of each line that continues to the next; the backslash and the newline following it
will be replaced with a single space.

\end{itemize}

\paragraph{Two more tricks} There are two more conveniences to the spec file syntax,
which can be skipped on a first reading.

First, the special key {\tt include} indicates that a separate file should be pasted in
where the {\tt include} key was given. For example, we may have a file named {\tt
checks.spec}:

\begin{lstlisting}[language=]
age < 15 && status='married'
age > 99
\end{lstlisting}

The main spec file would have a {\tt checks} section that looks like this:

\begin{lstlisting}[language=]
checks{
    include: checks.spec
}
\end{lstlisting}

The {\tt checks.spec} file would be pasted into place to reproduce the same {\tt checks}
section written in one place above. This mechanism allows one person to work on the
consistency rules in one file and another to work on the imputations in another file.

Second, smaller segments can also be pasted into place. This was originally intended to
minimize redundancy in the {\tt impute} segments below. The special {\tt paste in} key
will paste in a given set of key:values at the point given. Thus, this spec file:

\begin{lstlisting}[language=]
cats{
    categories{
        age
        sex
        tract
    }
}

impute {
    paste in: cats
    output var: income
    method: lognormal
}

impute {
    paste in: cats
    output var: sex
    method: hot deck
}
\end{lstlisting}

will expand to this more redundant spec file:

\begin{lstlisting}[language=]

impute {
    categories{
        age
        sex
        tract
    }

    output var: income
    method: lognormal
}

impute {
    categories{
        age
        sex
        tract
    }
    output var: sex
    method: hot deck
}
\end{lstlisting}

Notice that the key {\tt cats} gets used by {\tt paste in}, but does not appear in
the final spec file.


\subsection{Input declarations}
We now continue through our spec file to the {\tt input} key.

The {\tt input} key specifies:
\begin{enumerate}
\item The  CSV\footnote{The custom is to call these files {\em
    comma-separated values}, but you are not required to use commas as a delimiter.}
    file from which you will draw your data.
\item The option to \texttt{overwrite} the CSV file currently written into the specified database with 
a new CSV file that you have specified.
\item The {\tt output table} key that specifies the table where you would like to 
write the data that is read in from the {\tt CSV} file.
\end{enumerate}

More information about these keys can be found in the appendix.  A visual layout of
what the effects of the {\tt input} segment of the spec affects is shown in Figure
\ref{inputfigure}.

\begin{figure}
\begin{center}
\includegraphics[scale=.5]{dotInput.png}
\caption{input \{ $\ldots$ \} flow chart}\label{inputfigure}
\end{center}
\end{figure}

\subsection{Field declarations} The edit-checking system needs to know the type and range
of every variable that has an associated edit. If a variable does not have an associated
edit, there is no need to declare its range.

The declaration of the edit variables contained in the {\tt fields} key consists of the field name, 
a type ({\tt int}, {\tt cat}, or 
{\tt real}) to be discussed further below, and a list of valid values for integer and categorical variables. Here is a typical example:

\begin{lstlisting}[language=]
fields {
    age: int 0-100
    sex: cat M, F
    hh_type: int 1, 2, 4, 8
    income: real
}
\end{lstlisting}

The above code declared four variables: {\tt age}, {\tt sex}, {\tt hh\_type}, and
{\tt income} and we've declared those four variables in different and valid ways.
By declaring our variables with a type and range, we can pass the information to the
edit-checking system so that it knows what to verify when running its checks for each
of these variables.  You can see that the list of values may be text categories or
numeric values, and the range 0--100 will be converted into the full sequence 0, 1,
2, \dots, 100.

When declaring a variable, the first word following the {\tt :} must be a type. For instance, 
for the {\tt hh\_type: int 1, 2, 4, 8} field above, we used the word 'int' to indicate that 
the data values of the field were of type integer. If leading zeros are relevant (e.g.,
a category like {\tt 01}), specify the field as a text category to prevent the leading
zero from being removed.  Keep in mind 
that if you declare the incorrect type for a field that the edit-checking system may not correctly 
verify the values of that field in your data set.

As a final note, we warn the user against creating fields with an overly large range of
integer values. For a field with a range of possible integer values, the edit-checking
system will verify that each data point falls into one of the possible values specified
in the range. Though this is easily doable for smaller ranges such as 0-100 or even
0-1000, it becomes extremely time consuming for larger ranges such as 0-600000. Instead,
we recommend assigning a field with a large range as a real variable.


\begin{itemize}
\item Any variable used for consistency checking needs to have a key:value line in the
{\tt fields} section.
\item The value is of one of three forms:
    \begin{itemize}
    \item {\tt int <values>}
    \item {\tt cat <values>}
    \item {\tt real}
    \end{itemize}
\item The values for categories can be any list, separated by commas. Spaces before and
after commas are removed; spaces within category names are kept. 
\item Integer values are a comma separated list of numbers and ranges, such as {\tt -2, -1, 1-20, 20-30, 32}.
\end{itemize}

%Here is a visual representation of the {\tt fields} key:
%
%\begin{figure}[H]
%\begin{center}
%\includegraphics[scale=.5]{dotFields.png}
%\caption{fields \{ $\ldots$ \} key flow chart}\label{afigure}
%\end{center}
%\end{figure}

\subsection{Recodes}
In short, {\tt recode} keys are new variables that are deterministic functions of 
existing fields. Figure \ref{basicrecode} presents a typical example of a {\tt recodes} key:

\begin{figure}
\begin{lstlisting}[language=]
recodes {
    CATAGE {
        0 | age between 0 and 15
        1 | age between 16 and 64
        2 | age > 64
    }
}
\end{lstlisting}
\caption{A typical recode.}\label{basicrecode}
\end{figure}

Here, we have declared a new variable CATAGE whose data points are based on a
deterministic function of the variable AGEP. As we will see later, this recode will
be called in the {\tt categories} key during imputation so that the data points we
are attempting to impute will be imputed in categories based on their recode values
rather than collectively as a single set of data points.

Recodes are created via an SQL view, which is a virtual table that calculates the
values of recode fields on demand.

The name of the recode table is always the name of the input table preceded by {\tt
view}.  For example, the demo has an input table {\tt dc}, so subsequent steps use
the recode view {\tt viewdc} as input.

\begin{itemize}
\item The parsing of recodes is a little different from the other parsing. We may change
this in the future. 
\item Typical recodes have a name of the variable being created, followed by a sequence of
values of the form {\tt <value> | <formula>}.
\item The {\tt <value>} can be anything, but is typically a small integer.
\item The {\tt <formula>} is any SQL expression using existing variables; when the formula
is true, the variable takes on the associated value.
\item If a recode does not have a $|$, then it is interpreted as a plain SQL formula for
defining the recode.
\item There can be multiple {\tt recodes} segments. They are executed in sequence, so
later recodes can make use of variables set up in earlier recodes.

\end{itemize}

Figure \ref{recodefigure} is a visual representation of the {\tt recodes} key:

\begin{figure}
\begin{center}
\includegraphics[scale=.5]{dotRecodes.png}
\caption{records \{ $\ldots$ \} flow chart}\label{recodefigure}
\end{center}
\end{figure}

\subsection{Checks}
The consistency-checking system is rather complex, but this complexity is what makes
the system efficient and reliable: there are typically a few dozen to a few hundred
checks that every observation must pass, from sanity checks like {\em fail if age < 0}
to real-world constraints like {\em fail if age < 16 and status='married'}. Further,
every time a change is made (such as by the imputation system) we need to re-check that
the new value does not fail checks. For example, an OLS imputation of age could easily
generate negative age values, so the consistency checks are essential for verifying that
the imputation process is giving us accurate and usable data points. In addition to
error checking we can also use consistency checks for other purposes, such as setting
{\em structural zeros} for the raking procedure.

All of the checks that are to be performed are specified in the {\tt checks} key. 
The checks you specify here will be performed on all input variables, as well as on all 
imputed data values. Let's take a look at an example {\tt checks} key:

\begin{lstlisting}[language=]
checks {
    	age < 0
        age > 95 => age = 95
       }
\end{lstlisting}

This indicates that the consistency checking system should verify that age is not less
than 0 and that age is not greater than 95. Addtionaly we specify that a value higher
than 95 should simply be top-coded as 95. We have not included an automatic fix for
age < 0 because if a data-point has a negative age value than it's indicative of a
processing error in the incoming data.

Figure \ref{checkflow} is a visual representation of the {\tt checks} key and all of its subkeys.

\begin{figure}
\begin{center}
\includegraphics[scale=.5]{dotChecks.png}
\caption{checks \{ ... \} key flow chart}\label{checkflow}
\end{center}
\end{figure}


\subsection{Imputation} \label{imputesec}
The {\tt impute} key has several sub-keys that describe various steps in the imputation
process. Many of the values of these sub-keys are derived from values found earlier
in the spec file, e.g. {\tt categories} is based off of the variables declared in {\tt
recodes}. Here is an example of an {\tt impute} key that is used to
outline the imputation of the age variable described above:

\begin{lstlisting}[language=]
impute {
    input table: viewdc
    min group size: 3
    draw count: 3
    seed: 2332

    categories {
        CATAGE
        SEX
    }

    method: hot deck
    output vars: age
}
\end{lstlisting}

Let's walk through each of the sub-keys above and see what they're doing:
\begin{itemize}
\item The {\tt output table} sub-key specifies the name of a view table to 
where you will write your recode variables. 
\item {\tt min group size}\ indicates the minimum number of known data points that can be 
used as donor or training data to impute unknown data points. 
\item {\tt draw count}\ determines the number of multiple imputations that should be performed. 
Because imputation can be done using stochastic models, it is often beneficial to obtain multiple 
imputations of the same data set. This will be explained in more detail later. 
\item {\tt seed}\ specifies the pseudo-random number generator seed that is used to obtain 
random numbers for stochastic imputation models. Running Tea with the same seed will
always produce identical results, so it is good form to record the seed as part of the 
specification. Changing the random seed will produce a new stream of random draws.
\item {\tt categories}\  specifies which recodes will be used to impute the data set for 
the given variable. For example, if {\tt CATAGE} is listed in the {\tt categories} key then the 
data points that correspond to each of the three possible {\tt CATAGE} values will be imputed 
separately. While it's not an absolute requirement that the {\tt categories} key be implemented, most data 
sets are imputed by categories so you will be using this key for almost all of your imputations.
\item {\tt method}\ specifies the type of model that is used to impute your data points. 
Choosing this model is discussed more thoroughly later on in the tutorial.
\item {\tt output vars}\ specifies the variable whose data points are to be imputed. For this reason 
you could consider {\tt output vars} as both the `input' variable as well as the `output' variable 
to the model specified in {\tt method}.
\end{itemize}
More can be found on each of the above keys in the appendix.

\cmt{
Figure \ref{imputefigure} is a visual representation of the {\tt impute} key how the
elements are used.

\begin{figure}
\begin{center}
\includegraphics[scale=.5]{dotImpute.png}
\caption{impute \{ ... \} key flow chart.}\label{imputefigure}
\end{center}
\end{figure}
}

\subsection{More features}
This concludes our walkthrough of a typical spec file. By now you should have a basic idea of 
how a spec file is implemented within Tea. Before we continue on to explaining more about imputation, 
the models available in Tea, and other features that are available within the Tea framework, we will mention 
four more features of the spec file.

\paragraph{Group recodes} Tea does all consistency checking on a single observation by
itself. Thus, to make an edit like {\em people with relationship status of child must
be younger than the householder} work, the age of the householder must be on the same line
as the child's age.

Group recodes are much like individual recodes: they are a new variable that is a function
of an existing variable (perhaps from an earlier recode group in the spec file), but the
expression for generating the variable will be evaluated using all of the data in a
household. 

Figure \ref{morerecodes} shows a series of single-record recodes, then a set of group
recodes. The single-record recodes demonstrate the use of the SQL {\tt case} statement,
which is analogous to the if-then-else expressions from many other languages. You will
rarely (if ever) need this in individual recodes via Tea, because the list-of-categories
syntax shown in Figure \ref{basicrecode} simplifies writing out the cases.

The group recodes in Figure \ref{morecodes} will produce the given variables
by applying the given formula to every person in the survey with the same {\tt SERIALNO}
(one ACS household). Thus, each member of a household will have an {\tt NP} field listing
the total number of people in the household, an {\tt NHH} field listing the number of
people in the household who are householders, et cetera. 

The age of householder is the maximum age in the household only among those with {\tt RELP='00'}, which is expressed using 
a {\tt case} statement inside of the {\tt max()} function.

\begin{figure}
\begin{lstlisting}[language=]
recodes {
        MOVE: case MIG when 1 then 'moved' else 'stayed' end
        DEG: case when SCHL in ('24','23','22','21','20') then 'degree' else 'non-degreed' end
        MF: case SEX when '1' then 'Male' when '2' then 'Female' else NULL end
        REL: case RELP when '00' then 'Householder' when '01' then 'Spouse' \
                        when '02' then 'Child' \
                        when '06' then 'Parent' else NULL end
}

group recodes {
    group id: SERIALNO
    recodes {
        NP: max(SPORDER)
        NHH: sum(RELP='00')
        NSP: sum(RELP='01')
        NUP: sum(RELP='15')
        HHAGE: max(case RELP when '00' then AGEP end)
        SPAGE: max(case RELP when '01' then AGEP end)
        SPORD: max(case RELP when '01' then SPORDER end)
        HHSEX: cast(round(avg(case RELP when '00' then SEX end)) as integer)
        SPSEX: cast(round(avg(case RELP when '01' then SEX end)) as integer)
    }
}
\end{lstlisting}
\caption{Some more elaborate recodes}\label{morerecodes}
\end{figure}

\paragraph{Including} So far we've spoken of a single on-disk spec file that contains
all of the survey processing instructions.  Though this approach may be appropriate
in many scenarios, it's often the case that you and your colleagues would like to
make edits to the same spec file and update or construct it concurrently. To do this,
you can simply make subsidiary files to be included in the main spec file.

To include a subsidiary file\footnote{Note that any subsidiary files that you choose
to include in your spec file must be in the same directory as the spec file or it
will not be able to find them.} at a certain point in the spec file, simply insert
the key {\tt include: }\textit{subsidiary\_file\_name} at the line in the spec file
where you would like the contents of the subsidiary file to be inserted. For example,
if you have written the consistency checks in a file named {\tt consistency}, and the
entire rank swapping configuration was in a file named {\tt swap}, then you could use
this parent spec file:

\begin{lstlisting}[language=]
database: test.db

include: swap

checks {
    include: consistency
}
\end{lstlisting}

\paragraph{Flagging for disclosure} For a given crosstab, there may be cells with a
small number of people, perhaps only one. This feature of Tea allows us to flag those
cells for later handling.

Any combination of variables could be a crosstab to be checked, but flagging
typically focuses on only a few sensitive sets of variables. Here is the section
of the spec file describing the flagging. The {\tt key} list gives the variables
that will be crossed together. With {\tt combinations: 3}, every set of three
variables in the list will be tried. The {\tt frequency} variable indicates that
cells with two or fewer observations will be marked.

We are calling this specific form of disclosure avoidance {\em fingerprinting}, so after
this segment of the spec file is in place, call {\em doFingerprinting()} from R to run the
procedure. The output is currently in a database table named {\em vflags}.

\begin{lstlisting}[language=]
fingerprint {
    key{
        CATAGE
        SEX
        PUMA
        HISPF
        ESR
        PWGTP
    }
    frequency: 2
    combinations: 3
}
\end{lstlisting}


This concludes our discussion of the spec file layout and syntax. At this point, you should be 
able to implement a basic spec file to impute any data set. More information about the keys 
discussed above and others that were not present in {\tt demo.spec} can be found in the appendix and 
at the r-forge website.

\cmt{ %BK: All of raking is commented out. 
\paragraph{Raking} Each record in a data set might have several data items which require imputation for
different reasons:
\begin{itemize}
\item a record could have several inconsistent items that we must replace
\item an otherwise consistent record could have a blank item that we wish to impute.
\item a record could have a combination of consistent items that could lead to personal identification
\end{itemize}
Each scenario implies slightly different knowledge about the data, and thus each scenario
might require a different imputation method to properly use this knowledge.

An overlay is a secondary data table (or set of tables) that gives information regarding
the \emph{reason} for imputation.  Using missing data as an example, a simple overlay
could have an entry for each item in the data, indicating if that item is missing or not.
A more complicated overlay could delineate the type of non-response for each data item.

Raking is a method of producing a consistent table of individual cells beginning
with just the column and row totals. For this tutorial, we will use it as a
disclosure-avoidance technique for crosstabs. The column sums
and row sums are guaranteed to not change; all individual cells are
recalculated. Thus, provided the column totals have passed inspection for
avoiding disclosure, the full table passes.

The key inputs are the set of fields that are going to appear in the final
crosstab, and a list of sets of fields whose column totals (and cross-totals)
must not change.

In the spec file, you will see a specification for a three-way crosstab between
the {\tt PUMA}, {\tt rac1p}, and {\tt catage} variables. All pairwise crosstabs
must not change, but any other details are free to be changed for the raking.

\begin{lstlisting}[language=]
raking {
    all_vars: puma|catage | rac1p

    contrasts{
        catage | rac1p
        puma | rac1p
        puma | catage
    }
}
\end{lstlisting}

As you have seen a few times to this point, once you have the spec file in place you can
call the procedure with one R function, which in this case is {\tt doRaking()}. But there
are several ways to change the settings as you go, so that you can experiment and adjust.

The first is to simply change the spec file and re-run. To do this, you will have to call
{\tt read\_spec} again:
\begin{lstlisting}[language=]
> read_spec("demo.spec"); doRaking()    #Run two commands on a line by ending the first with a semicolon
> read_spec("demo.spec"); doRaking()    #Hit the up-arrow to call up the previous line.
\end{lstlisting}

Everything you can put in a spec file you can put on the command line. The help system
will give you the list of inputs (which will also help with writing a spec file), and you
can use those to tweak settings on the command line:
\begin{lstlisting}[language=]
> ?doRaking                   #What settings are available?
> doRaking(max_iterations=2)  #What happens when the algorithm doesn't have time to
> converge?
\end{lstlisting}
} %end commented-out raking section.


\section{Interactive R}\label{rsec}
The specification file described to this point does nothing by itself, but provides
information to procedures written in R that do the work. In fact, Tea is simply 
called as a library in R, and the actions described in the spec file are executed by
issuing commands from the R command prompt described below.

\subsection{Loading Tea in R}
When you start R (from the directory where the data is located), you are left at a
simple command prompt, $>$, waiting for your input. Tea extends R via a library of
functions for survey processing, but you will first need to load the library, with:
\begin{lstlisting}[language=]
> library(Tea)
\end{lstlisting}

Now you have all of the usual commands from R, plus those from Tea. You only
need to load the library once per R session, but it's harmless if you run
{\tt library(Tea)} several times.

After loading the library by running {\tt library(tea)}, you would then need to 
tell R to read your spec file, perform the checks, perform the imputations, 
and then finally check out the imputations to an R data structure so that you 
can view them. Observe the following code:

\begin{lstlisting}[language=]
> library(tea)
> readSpec("spec")
> doChecks()
> doMImpute()
> checkOutImpute("fulldata")
\end{lstlisting}

These commands could be entered in a script file as easily as they are entered on R's command line.
You always have the option of creating a {\tt .R} file that has each of the above commands listed 
sequentially. Observe the following example of a {\tt .R} file:

\begin{lstlisting}[language=]
library(Tea)
library(ggplot2)
readSpec("demo.spec")
doChecks()
doMImpute()
checkOutImpute()
\end{lstlisting}

If we assume that the file above is named {\tt demo.R} then we could run the following command 
from R's command line:

\begin{lstlisting}[language=]
> source("demo.R")
\end{lstlisting}

Starting R and sourcing a script allows you to further query and interrogate the data
after execution. To run {\tt demo.R} without starting R's interactive shell, use {\tt R -f
demo.R}.

In either case, once your spec file has been correctly implemented, your data will
be imputed and available for viewing through an R data frame.  We examine how this is
done in the next subsection.

\subsection{Showing the data}
Data is stored in two places: the database, and R data frames. Database tables
live on the hard drive and can easily be sent to colleagues or stored in
backups. R data frames are kept in R's memory, and so are easy to manipulate and
view, but are not to be considered permanent. Database tables can be as large as
the disk drive can hold; R data frames are held in memory and can clog up memory if
they are especially large.

You can use the {\tt teaTable} function to pull a part of a database table into
an R data frame. You probably don't want to see the whole table, so there are
various options to limit what you get. Some examples:
\begin{lstlisting}[language=]
> TeaTable("dc", cols="AGEP, PUMA", where="PUMA=104")
> TeaTable("dc", cols="AGEP, PUMA", limit=30, offset=100)
\end{lstlisting}

The first example pulls two columns, but only where {\tt PUMA=104}. The second
example pulls 30 rows, but with an offset of 100 down from the top of the
table. You will probably be using the {\tt limit} often; the {\tt offset} allows
you to check the middle of the table, if you suspect that the top of the table
is not relevant or representative of the data need to analyze.

There are, of course many more options than those listed here. If you'd like to see what other tools are available in manipulating R data, you can get them via R's help system as such:
\begin{lstlisting}[language=]
> ?TeaTable
\end{lstlisting}

This {\tt ?name} form should give you a help page for any function, including Tea
functions like {\tt ?doRaking} or {\tt ?doMImpute}. (Yes, Tea function documentation is still a
little hit-and-miss.) Depending on R's setup, this may start a paging program that lets 
you use the arrow keys and page up/page down keys to read what could be a long document. 
Quit the pager with {\tt q}.

The {\tt show\_db\_table} function creates an R data frame, and, because the examples above didn't do
anything else with it, displays the frame to the screen and then throws it out. Alternatively, you can
save the frame and give it a name. R does assignment via {\tt $<$-}, so name the output
with:

\begin{lstlisting}[language=]
> p104 <- TeaTable("dc", cols="AGEP, PUMA", where="PUMA=104")
\end{lstlisting}

To display the data frame as is, just give its name at the command line.

\begin{lstlisting}[language=]
> p104
\end{lstlisting}

But you may want to restrict it further, and R gives you rather extensive control over
which rows and columns you would like to see.

In both R and the spec file syntax, a {\tt \#} indicates a comment to the human
reader; everything from the {\tt \#} to the end of the line is ignored by the parser.

\begin{lstlisting}[language=]
> p104[1,1]          #The upper-left element
> p104[1,"AGEP"]     #The upper-left element, using the column name
> p104[ ,"AGEP"]     #With no restriction on the rows, give all rows---the full AGEP vector
> p104[17, ]         #All of row 17

> minors <- p104[, "AGEP"] < 18
> minors             #A true/false vector showing which rows are under 18.
> p104[minors, ]     #You can use that list of true/falses to pick rows. This gives all rows under 18yo.
\end{lstlisting}

These commands could be entered on R's command line as easily as in a script file, so an
analyst who needs to verify the results of the consistency-checking step could copy and
paste the first three lines of the script onto the R command prompt, where they will run and
then return the analyst to the command prompt, where he or she could print subsections of
the output tables, check values, modify the spec files and re-run, continue to the
imputation step, et cetera.


\section{Imputation}\ref{impsec}
Thus far we've seen how to impute a data set using a single {\tt impute} group. In this form, single 
imputation produces a list of replacement values for those data items that fail consistency checks 
or initially had no values. For the case of editing, the replacements would be for data that fails 
consistency checks; for the case of basic imputation, the replacements would be for elements that
initially had no values. Recall that as we stipulated earlier in the tutorial, for our purposes we 
consider a looser definition of imputation that includes the replacement of data points that both 
initially had no value as well as those that fail consistency checks.

In a similar vein, while single imputation gives us a single list of replacement values, any stochastic 
imputation method could be repeated to produce multiple lists of replacement values. For instance, 
we could use multiple imputation to calculate the variance of a given statistic, such as average 
income for a subgroup, as the sum of within-imputation variance and across-imputation variance.

To give a concrete example, consider a randomized hot deck routine, where missing values are
filled in by pulling values from similar records. For each record with a missing income:
\begin{enumerate}
\tighten
\item Find the universe of which the record is a member.
\item For each variable in turn:
    \begin{itemize}
    \item Make a draw from model chosen for the variable, based on the specific universe from (1).
    \end{itemize}
\end{enumerate}


The simplest and most common example is the randomized hot deck, in which each survey respondent 
has a universe of other respondents whose data can be used to fill in the respondent's missing
values. The hot deck model is a simple random draw from the given universe for the given variable.

Given this framework, there are a wealth of means by which universes are formed, and
a wealth of models by which a missing value can be filled in using the data in the chosen universe.

\paragraph{Universes}
The various models described above are typically fit not for the whole data set, but for
smaller {\em universes}, such as a single county, or all males in a given age group.
A universe definition is an assertion that the variable set for the records in the
universe was generated in a different manner than the variables for other universes.

An assertion that two universes have different generative processes could be construed in
several ways:

\begin{itemize}\tighten
\item different mathematical forms, like CART vs.\ logistic regression
\item One broad form, like logistic regression, but with different variables chosen
    (that is, the stochastic form is the same but the structural form differs).
\item A unified form, like a logistic regression with age, sex, and ancestry
    as predictors, with different parameter estimates in each universe.
\end{itemize}

Universe definitions play a central role in the current ACS edit and imputation system.
Here, universes allow data analysts to more easily specify particular courses of
action in the case of missing or edit-inconsistent data items. To give an extreme example, for the imputation of marital status in ACS group quarters (2008), respondents are placed
in two major universes: less than 15 years of age (U1) and 15 or more years of age
(U2). The assertion here, thus, is that people older than 15 have a marital status
that comes from a different generative process than those people younger than 15.
This is true: people younger than 15 years of age cannot be married!  Thus in the
system, any missing value of marital status in U1 can be set to ``never married'',
and missing values in U2 can be allocated via the ACS hot-deck imputation system.

Now that we are more familiar with universes, we can discuss the various models that are 
available in Tea and the method of choosing the one that is appropriate for the 
imputation being performed.


\section{Models}
Now that we're more familiar with the concept of universes, we examine how to choose the 
model that will give us the best results when imputing the data points in a specific universe.
Indeed, given an observation with a missing data point and a universe, however specified, there is
the problem of using the universe to fill in a value. Randomized hot-deck is again
the simplest model: simply randomly select an observation from the universe of acceptable
values and fill in. Other models make a stronger effort to find a somehow-optimal value:\\

$\bullet$ One could find the nearest neighbor to the data point to be imputed (using any
of a number of metrics).\\

$\bullet$ Income is typically log-Normally distributed, and log of this year's income, last
year's income, and if available, income reports from administrative records (\adrec) may
be modeled as Multivariate Normal (with a high correlation
coefficient among variables). After estimating the appropriate Multivariate
distribution for the universe, one could make draws from the distribution.\\

$\bullet$ If the Multivariate Normal seems implausible, one could simply aggregate the
data into an empirical multivariate PDF, then smooth the data into a kernel
density estimator (KDE). Draws from a specified KDE can be made as easily as
draws from a standard Multivariate Normal.\\

$\bullet$ Count data, such as the number of hospital visits over a given period, are
typically modeled as a Poisson distribution. In a manner similar to fitting a Normal, one
could find the best-fitting Poisson distribution and draw from that distribution to fill
in the missing observation.\\

$\bullet$ Bayesian methods: if \adrec are available, they may be used to generate a prior distribution
on a variable, which is then updated using non-missing data from the survey.\\

$\bullet$ One could do a simple regression using the data on hand. For example,
within the universe, regress income on  age, race, sex, and covariates from \adrec.
Once the regression coefficients are calculated, use them to impute income for
the record as the predicted value plus an error term. \cmt{ Note that OLS assumes
a Normally-distributed error term and therefore gives a distribution of values for
the predicted income, not just a single number; therefore one could draw multiple
imputations.}\\

$\bullet$ Discrete-outcome models, like the Probit or Logit, could be used in a similar
manner.\\

$\bullet$ To expand upon running a single regression, one can conduct a structured search
over regression models for the best-fitting regression.\footnote{Because the imputation
step is not an inference step, such a search presents few conceptual difficulties.}
Such a search is currently in use for disclosure avoidance in ACS group quarters.\\

A unified framework would allow comparison across the various imputation
schemes and structured tests of the relative merits of each. Though different surveys
are likely to use different models for step (2) of the above process, the
remainder of the overall routine would not need to be rewritten across surveys.

We now discuss each of the models that are available in Tea.

\cmt{DV: We need to add rel to this list.}
\subsection{Models of Tea} This section describes the models available for use in imputing
missing data.

\paragraph{Hot deck} This is randomized hot deck. Missing values are filled in by
drawing from nonmissing values in the same subset. The assumption underlying the model
is {\em missing completely at random} (MCAR), basically meaning that nonrespondents
do not differ in a systematic way from respondents.

This model has no additional keys or options, although the user will probably want an
extensive set of category subsets. Example:

\begin{lstlisting}
impute{
    input table: dc
    min group size: 3
    draw count: 5
    id: serialno
   categories {
        agecat
        sex
    }
    output vars: sex
    method: hot deck
}

impute{
    input table: dc
    min group size: 3
    draw count: 5
    id: serialno

    categories {
        agecat
        sex
    }
    method: ols
    output vars: agep
    input vars: rac1p, nativity||sex
}
\end{lstlisting}

\paragraph{Ordinary least squares (aka regression)}  This is the familiar $y = \beta_0
+ \beta_1 x_1  + \beta_2 x_2 + \dots +\epsilon$ form. The parameters (including the
variance of $\epsilon$) are estimated for the subset of the category where all variables
used in the regression are not missing. For a point where $y$ is missing but $(x_1, x_2, \dots)$ are not,find $\beta_1 x_1  + \beta_2 x_2 + \dots$, then add a random draw from the distribution of $\epsilon$.

The variables may be specified via the usual SQL, with two exceptions to accommodate the
fact that so many survey variables are categorical.

\cmt{%BK: Not sure how much of this still works.
Unless otherwise noted, all dependent variables are taken to be categorical, and so are expanded to
a set of dummies. The first category is taken to be the numeraire, and others are broken
down into dummy variables that each have a separate term in the regression. The
independent variable will always be calculated as a real number, but depending on the type
of variable may be rounded to an integer.

If a dependent variable is numeric, list it with a \#, such as {\tt variables:
\#age, sex}.

An {\em interaction} term is the product of the variables, where for catgories {\em
product} means the smaller subsets generated by the cross of the two variables, such as
a sex-age cross of $(M, 0--18), (F, 0--18), (M, 18--35), (F, 18--35)$; for continuous
variables {\em product} has its typical meaning.
}

\cmt{
Because every dependent variable must be complete for the regression to calculate the
independent variable, it makes sense to put regressions in a sequence, with a spec file like:

\begin{lstlisting}
impute (first){
        models{
        rac1p { method: hot deck }
        nativity { method: hot deck }
        sex { method: hot deck }
        }
}

impute (second){
        models{
        agep { method: ols
               vars:  rac1p, nativity||sex
        }
        }
}
\end{lstlisting}

and then two steps in R:

\begin{lstlisting}
doMImpute(first)
doMImpute(second)
\end{lstlisting}
}

\paragraph{Probit and Logit} These models behave like OLS, but are aimed at categorical
variables. The output variable is not restricted to twoc categories.

%Need to fill in this paragraph
\paragraph{seqRegAIC}


\paragraph{Distributions} The Normal (aka Gaussian) distribution is the archetype of this
type of model. First, estimate the mean and standard deviation using the non-missing data
in the category. Then fill in the missing data via random draws from the Normal
distribution with the calculated parameters. You may use either {\tt method: normal} or
{\tt method: gaussian}.

For other situations, other distributions may be preferable. For example, income is
typically modeled via {\tt method: lognormal}. Count data may best be modeled via {\tt
method: poisson}.

Hot deck is actually a fitting of the Multinomial distribution, in which each bin has
elements in proportion to that observed in the data; {\tt method: hot deck} and {\tt
method: multinomial} are synonyms.

The {\em method: multivariate normal} doesn't work yet.

The distribution models have no additional options or keys.

\paragraph{Kernel smoothing} A kernel density estimate overlays a Normal distribution over
every data point. For example, if a data set consisted of two data points at 10 and one at
12, then there would be a large hump in the final distribution at 10, a smaller hump at
12, and the distribution would be continuous over the entire space of reals.

Thus, kernel smoothing will turn a discrete distribution consisting of values on a
few values into a continuous distribution.

Invoke this model using either {\tt method: kernel} or {\tt method: kernel density}.

\todo{We have written but not incorporated an HTML viewer for data using overlay
information.}

\paragraph{Internals}
The Census Bureau often deals with tables that would have 1.5 million cells if written
out in full, but where only maybe 20,000 surveys are in hand. Therefore, the algorithm
is heavily oriented toward sparse data.

It begins with a long and tedious routine to write SQL to generate the set of
possibly-nonzero values, as per the example above. SQL is the appropriate language
for generating this list because it is optimized for generating the cross of several
variables and for pruning out values that match our criteria. The tedium turns out
to be worth it: our test data set takes about 25 seconds to run using the original
full-cross `72 algorithm, and runs in under two seconds using the SQL-pruned matrix.

\cmt{
\section{Synthesis}
Group quarters data in the 2010 Decennial census will be protected via
a method known as ``synthetic data''. This method uses statistical models to modify
records in need of confidentiality protection. Such a method has already been applied to
group quarters records in the American Community Survey for collection years 2006--2009.

\section{Weighting}
We use the {\em survey} package for R to calculate and modify weights for a
survey. The package is fully documented in \citet{lumley:surveys}.

\section{Models and Universes}
Statistical models are, in essence, an assertion about the \emph{process} that generates
the data. As such, useful models for imputation are those that imply processes that
we believe generate ACS GQ data items. Models are typically prescribed at the level
of variables; that is, we seek to describe the generative process of a set of variables
for a certain population.

A given variable set could have several distinct processes by which its values
are generated. In this case, it would be suboptimal to use only a single model to
perform imputation for this variable set. We group respondent records together into a
``universe'' when we believe their values for a given set of variables are generated
by the same process. Different sets of variables that require imputation could thus
require different universe definitions.

\subsection{Models}
Here, we list several models used to predict or fill in a missing element of
a record. The element could be missing because it was not reported (imputation),
because it was incorrectly reported (editing), or because leaving the correct value
as it lays would create disclosure risk. For any of these cases, any of the models to
follow could be applied. Note that these models are typically inserted into a larger
procedure; for example, a series of models could be used to fill in various data items.

{\bf Note on MI: to be ``proper'' MI models, they would need to produce
posterior predictive densities, so we might mention that deterministic
procedures don't fit the bill}

\subsubsection{Randomized hot deck/multinomial} This is a univariate method, that
randomly draws a value from the list of nonmissing values for the variable.
{\bf you can have a multivariate hot deck: we could get a value of A and B
from a cell defined by C and D}

\subsubsection{Nearest-neighbor} Based upon some metric defined by the user, find the
record closest to the record to be filled in, and copy the value(s) from that
record.

\subsubsection{Probabilistic nearest-neighbor} Define the odds of selecting a record
as inversely proportional to the record's distance to the record to be filled
in. That is, the nearest neighbor is the most likely to be selected, but others
may also be selected. After selecting a record, copy the value(s) as with
the deterministic nearest-neighbor model.

\subsubsection{Linear models} Given a complete-data subset, fit a generalized linear
model (GLM) to be specified by the user. The predictors used for a model could
be selected by hand or could be chosen automatically based on a given
criterion, such as the Akaike/Bayesian Information Criterion.
{\bf we could also suggest sequential models that start of predicting based
upon complete items; that is, if V1 is complete in the subset, first do
V2 \~{} V1, then V3 \~{} V1 + V2, etc; if nothing in complete, we could
always pick one to fill in randomly (that is, just use a prior)}

}

Recall that we had briefly discussed multiple imputation in the {\tt impute} section above. 
We now discuss this in more detail.

\subsection{Multiple Imputation}
A single imputation would produce a list of replacement
values for certain data items. Any stochastic imputation method could be repeated
to produce multiple lists of replacement values. Variance of a given statistic,
such as average income for a subgroup, is then the sum of within-imputation variance and
across-imputation variance.

\begin{itemize}
\tighten
\item For each imputation:
    \begin{itemize}
\tighten
    \item fill in the data set using the given set of imputed values
    \item calculate the within-imputation variance
    \end{itemize}
\item Sum the across-imputation and average within-imputation variances to
produce an overall variance figure that takes into account uncertainty due to imputation.
\end{itemize}

The question of what should be reported to the public from a sequence of imputations
remains open. The more extensive option would be to provide multiple data sets; the less
extensive option would be to simply provide a variance measure for each data point that
is not a direct observation.

\cmt{
	DV: The two paragraphs below (and more specifically the figures in them) probably need 
	to be cleaned up. I got rid of the \listing{ ... } tags below that were around the 
	\lstlisting{...} tags because they weren't getting compiled correctly but if you 
	want to put them in the old code that originally had them can be found in 
	/Tea/doc/archived_tutorial_documents/tutorial.tex}
        \paragraph{Interface} Figure \ref{sippconfig} shows a (slightly abbreviated)
        configuration file describing the hot deck process for a variable in the SIPP.
        It is intended to be reasonably readable, and maintainable by an analyst who is
        a statistician but not a programmer.

        Lines 11--18 of the configuration specify the
        categories used for step (1) above: draws are made from the universe of records with an
        age in the same age category as the record to be filled in and {\tt num\_sipp\_jobs\_2008} in the same
        category as well.

        Of course, different surveys would have different classification schemes, but
        this means that each survey would need a new configuration file, not new code.

        Line eight indicates that five imputations are to be done for each missing
        value. Those with extensive experience with multiple imputation often advise
        that a handful of imputations are sufficient for most purposes.

        The sample from
        Figure \ref{sippconfig} focused on the determination of categories in which to
        do imputation. Figure \ref{acsconfig} focuses on regression models that go
        beyond the simple randomized hot deck of Figure \ref{sippconfig}. Lines 5--8
        specify the variables that need imputation, and the form of model to be used.
        The current system will search the set of models of the given form for the one
        that best fits the known data; Lines 9--14 show the list of variables that could be
        used as explanatory variables, although a typical model will likely wind up
        using only around four or five.

     \paragraph{Edits} The system as written includes a component that checks consistency
        against a sequence of edits. In line three of the sample spec file of Figure
        \ref{sippconfig}, the {\tt flagearn} variable is declared to have possible values of 0,
        1, 3, or 4, but line four specifies that if an imputation returns 4, then it is rejected.
        The imputation routine sketched above does not need to include any edit rules, because
        this edit step will take those into account; the separation of edits and imputations
        simplifies the routine.
        \begin{lstlisting}[language=,numbers=left,numberstyle=\scshape]
        database: sipp.db

        |flagearn 0, 1, 3, 4
        flagearn = 4

        impute_by_groups {
            min_group_size  {20}
            iteration_count  {5}
            datatab  {sippdata}

            categories {
               15<=agesipp200812<18;
               18<=agesipp200812<22;
               22<=agesipp200812<40;
               40<=agesipp200812<62;
               62<=agesipp200812;
               num_sipp_jobs_2008 = 0;
               num_sipp_jobs_2008 = 1;
               num_sipp_jobs_2008 => 2;
         }

            imputes{
               flagearn~ flagearn;
            };
        }
        \end{lstlisting}

        \begin{lstlisting}[language=,numbers=left,numberstyle=\scshape]
        database: acs2008.db

        impute{
            seqRegAIC{
                vars{
                  TI{ model: gam }
                    DIS{ model: multinom }
                }
                predlist{ #Sample of predictors that could used for regressions
                    SEX; YOE; WKL; MIL; UR;
                    SCH; RCGP; POV; SS; MAR;
                    LANX; JWTR; TYPGRP; GQINST; FER;
                    ESR; DIS; COW; CIT; OCC2;
                }
            }
        }

        \end{lstlisting}


\section{Conclusion}
This concludes our tutorial. If you have any questions or comments please feel
free to contact Ben Klemens (ben.klemens@census.gov) or Rolando Rodr\'iguez
(rolando.a.rodriguez@gmail.com). Also, check out the R-Forge Website at
\url{https://r-forge.r-project.org/projects/Tea/} for updates to both the Tea software
as well as this tutorial document.

\section*{Appendix: keys}

This is a reference list of all of the available keys that could appear in a spec file.
As a reference, descriptions are brief and assume you already know the narrative of the
relevant procedures, in the main text.

Keys are listed using the {\tt group/key/value} notation described in Section \ref{specsec}.

%\input ../keys.tex

%\bibliographystyle{harvard}
%\bibliographystyle{plainnat}
%\bibliography{Tea}
\end{document}
