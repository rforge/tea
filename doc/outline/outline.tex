\documentstyle{article}                                                                                                                 
                                                                                                                                        
\begin{document}                                                                                                                        
\title{A two-session day course on TEA and R}                                                                                                   
\abstract{This two-session course is aimed at people familiar with Census operations using SAS. Part I covers some basic concepts in editing and multiple imputation; Part II applies them using a new R package, named TEA, that handles several post-processing steps. }                                                                                                                                       
         
\long\def\ii#1{\begin{enumerate}\item #1\end{enumerate}} 

\ii{Part I: Concepts
\item Imputation
	\item Models
    \ii{Donor-based
		\ii{Hot-Deck
		\item Predictive Matching}
	}
	\ii{Draw-based
    	\ii{Univariate
    	\item Conditional
    	\item Multivariate
		}
	}

\item Environments
	\ii{Editing
	\item Missing Data
	\item Disclosure
	}
\item Multiple Imputation


\item Part II: Application
\item R basics
	\ii{Startup; welcome to the cmd prompt
            \item Some SQL in here?
             }
	\ii{Tea basics
	\item spec file run-thru
	   
            }
 

}
                                                                                                                                        
\ii {Editing
\item SQL (via RSQLite)
    \ii{Basic {\tt select}  queries
    \item {\tt where}, {\tt limit}, {\tt order by}
    \item joins (and indexes)
    }                                                                                                                                   
\item Editing
    \ii{Some theory by Bill?}                                                                                                                                   
\item TEA
    \ii {A walk through a spec file
    \item imputation methods
    }
}                                                                                                                                       
                                                                                                                                        
\end{document}
