\documentstyle{article}                                                                                                                 
                                                                                                                                        
\begin{document}                                                                                                                        
\title{A two-session day course on TEA and R}                                                                                                   
\abstract{This two-session course is aimed at people familiar with Census operations using SAS. Part I covers some basic concepts in editing and multiple imputation; Part II applies them using a new R package, named TEA, that handles several post-processing steps. }                                                                                                                                       
         
\long\def\ii#1{\begin{enumerate}\item #1\end{enumerate}} 

\ii{Philosophy
	\ii{Imputation $=$ Data$\rightarrow$ Model$\rightarrow$ Fit$\rightarrow$ Draw$\rightarrow$ Data
	\item Methods $=$ Collections of imputations
	%communicative (update then re-model) e.g. SRMI 
	%non-communicative e.g. subsetting
	%combination e.g. SRMI on subsets
	\item Multiple Imputation
	}
\item Models
	\ii{Explicit
		\ii{Univariate
		\item Regression
		\item Multivariate
		}
	\item Implicit
		\ii{Classification
		\item Matching
		}
	}
\item Methods
	\ii{Donor-based
		\ii{Hot-Deck
		\item Predictive Matching}
	\item Density-based
		\ii{Predictive Distributions
		\item Sequential Regression
		\item Bootstrapping
		}
	}
\item Environments
	\ii{Editing
	\item Missing Data
	\item Disclosure
	}



\item Part II: Application
\item R basics
	\ii{Startup; welcome to the cmd prompt
            \item Some SQL in here?
             }
	\ii{Tea basics
	\item spec file run-thru
	   
            }
 

\ii {Editing
\item SQL (via RSQLite)
    \ii{Basic {\tt select}  queries
    \item {\tt where}, {\tt limit}, {\tt order by}
    \item joins (and indexes)
    }                                                                                                                                   
\item Editing
    \ii{Some theory by Bill?}                                                                                                                                   
\item TEA
    \ii {A walk through a spec file
    \item imputation methods
    }
}
}                                                                                                                                       
                                                                                                                                        
\end{document}
