\begin{key}{raking/thread count}
 You can thread either on the R side among several tables,
     or interally to one table raking. To thread a single raking process, set this to the
     number of desired threads. 
     
\end{key}

\begin{key}{input/primary key}
 A list of variables to use as the primary key for the output table.
In SQLite, if there is only one variable in the list as it is defined as an integer,
this will create an integer primary key and will thus be identical to the auto-generated
ROWID variable.
\end{key}

\begin{key}{group recodes/recodes}
 A set of recodes like the main set, but each calculation of the recode will be grouped by the group id, so you can use things like {\tt max(age)} or {\tt sum(income)}.
Returns 0 on OK, 1 on error.

\end{key}

\begin{key}{    raking/tolerance}
     If the max(change in cell value) from one step to the next
       is smaller than this value, stop. 
\end{key}

\begin{key}{input/types}
 A list of \emph{keys} of the form:
var: type
where var is the name of a variable (column) in the output table and type is a valid
database type or affinity.  The default is to read in all variables as character
columns.
\end{key}

\begin{key}{id}
 Provides a column in the data set that provides a unique identifier for each
observation.
Some procedures need such a column; e.g., multiple imputation will store imputations in a
table separate from the main dataset, and will require a means of putting imputations in
their proper place. Other elements of TEA, like flagging for disclosure avoidance, use the
same identifier.
\end{key}

\begin{key}{rankSwap/max change}
 maximal absolute change in value of x allowed.
That is, if the swap value for $x_i$ is $y$, if $|y - x_i| >$ maxchange,
then the swap is rejected

default = 1
\end{key}

\begin{key}{    raking/all vars}
     The full list of variables that will be involved in the
       raking. All others are ignored. 
\end{key}

\begin{key}{  impute/draw count}
   How many multiple imputations should we do? Default: 5.
\end{key}

\begin{key}{rankSwap/seed}
 The random number generator seed for the rank swapping setup.
\end{key}

\begin{key}{  impute/earlier output table}
   If this imputaiton depends on a previous one, then give the fill-in table from the previous output here.
 
\end{key}

\begin{key}{impute/input table}
 The table holding the base data, with missing values. 
  Optional; if missing, then I rely on the sytem having an active table already recorded. So if you've already called {\tt doInput()} in R, for example, I can pick up that the output from that routine (which may be a view, not the table itself) is the input to this one. 
\end{key}

\begin{key}{  impute/output table}
   Where the fill-ins will be written. You'll still need {\tt checkOutImpute} to produce a completed table.
\end{key}

\begin{key}{  impute/seed}
   The RNG seed
\end{key}

\begin{key}{raking/contrasts}
 The sets of dimensions whose column/row/cross totals must be kept constant. One contrast to a row; pipes separating variables on one row.
\begin{lstlisting}
raking{
    contrasts{
        age | sex | race
        age | block
    }
}
\end{lstlisting} 
\end{key}

\begin{key}{input/indices}
 Each row specifies another column of data that needs an index. Generally, if you expect to select a subset of the data via some column, or join to tables using a column, then give that column an index. The {\tt id} column you specified at the head of your spec file is always indexed, so listing it here has no effect.
\end{key}

\begin{key}{rankSwap/swap range}
 proportion of ranks to use for swapping interval, that is
if current rank is r, swap possible from r+1 to r+floor(swaprange*length(x))

default = 0.5

\end{key}

\begin{key}{    raking/count col}
     If this key is not present take each row to be a
		single observation, and count them up to produce the cell counts to which the
		system will be raking. If this key is present, then this column in the data set
		will be used as the cell count.
\end{key}

\begin{key}{input/input file}
 The text file from which to read the data set. This should be in
the usal comma-separated format with the first row listng column names.
\end{key}

\begin{key}{recodes}
 New variables that are deterministic functions of the existing data sets.
There are two forms, one aimed at recodes that indicate a list of categories, and one
aimed at recodes that are a direct calculation from the existing fields.
For example (using a popular rule that you shouldn't date anybody who is younger than
(your age)/2 +7),
\begin{lstlisting}[language=]
recodes { 
    pants {
        yes | leg_count = 2
        no  |                   #Always include one blank default category at the end.
    }
    youngest_date {
        age/2 + 7
    }
}
\end{lstlisting}
You may chain recode groups, meaning that recodes may be based on previous recodes. Tagged
recode groups are done in the sequence in which they appear in the file. [Because the
order of the file determines order of execution, the tags you assign are irrelevant, but
I still need distinct tags to keep the groups distinct in my bookkeeping.]
\begin{lstlisting}
recodes [first] {
    youngest_date: (age/7) +7        #for one-line expressions, you can use a colon.
    oldest_date: (age -7) *2
}
recodes [second] {
    age_gap {
        yes | spouse_age > youngest_date && spouse_age < oldest_date
        no  | 
    }
}
\end{lstlisting}
If you have edits based on a formula, then I'm not smart enough to set up the edit table
from just the recode formula. Please add the new field and its valid values in the \c
fields section, as with the usual variables.
If you have edits based on category-style recodes, I auto-declare those, because the
recode can only take on the values that you wrote down here.
\end{key}

\begin{key}{raking/input table}
 The table to be raked. 
\end{key}

\begin{key}{input/missing marker}
 How your text file indicates missing data. Popular choices include "NA", ".", "NaN", "N/A", et cetera.

\end{key}

\begin{key}{timeout}
 Once it has been established that a record has failed a consistency
   check, the search for alternatives begins. Say that variables one, two, and three each have 100
    options; then there are 1,000,000 options to check against possibly thousands
    of checks. If a timeout is present in the spec (outside of all groups), then the
    alternative search halts and returns what it has after the given number of seconds
    have passed.
   
\end{key}

\begin{key}{    raking/max iterations}
     If convergence to the desired tolerance isn't 
       achieved by this many iterations, stop with a warning. 
\end{key}

\begin{key}{input/output table}
 The name of the table in the database to which to write the data read in.
\end{key}

\begin{key}{database}
 The database to use for all of this. It must be the first thing on your line.
I need it to know where to write all the keys to come.
\end{key}

\begin{key}{    raking/run number}
     If running several raking processes simultaneously via
        threading on the R side, specify a separate run\_number for each. If
        single-threading (or if not sure), ignore this.
     
\end{key}

\begin{key}{input/overwrite}
 If {\tt n} or {\tt no}, I will skip the input step if the output table already exists. This makes it easy to re-run a script and only sit through the input step the first time.
\end{key}

\begin{key}{group recodes}
 Much like recodes (qv), but for variables set within a group, like
eldest in household.
For example,
\begin{lstlisting}[language=]
group recodes { 
    group id : hh_id
    eldest: max(age)
    youngest: min(age)
    household_size: count(*)
    total_income: sum(income)
    mean_income: avg(income)
}
\end{lstlisting}

\end{key}

\begin{key}{raking/structural zeros}
 A list of cells that must always be zero, 
     in the form of SQL statements. 
     
\end{key}

\begin{key}{input/primarky key}
 The name of the column to act as the primary key. Unlike other indices, the primary key has to be set on input.
\end{key}

\begin{key}{group recodes/group id}
 The column with a unique ID for each group (e.g., household number).
\end{key}

